\documentclass{beamer}

\usetheme{Frankfurt}
\usecolortheme{seahorse}
\usefonttheme{default} 
\setbeamertemplate{navigation symbols}{} % Quita los símbolos de navegación


\usepackage[utf8]{inputenc}
\usepackage[spanish]{babel}
\usepackage{amsmath}
\usepackage{amsfonts}
\usepackage{amssymb}
\usepackage{graphicx}
\graphicspath{{Graficos/Curso-LaTeX/Graficos/graficos/}} % Ruta de las imágenes
\usepackage{tikz}


\title[Presentaciones]{Presentaciones}
\subtitle{Aspectos Clave para Crear y Defender Presentaciones}
\author{José de los Ríos}
\institute{Universidad de Granada}
\date{\today}
\logo{\includegraphics[width=1.5cm]{Graficos/Curso-LaTeX/Graficos/graficos/ugr.png}}

\begin{document}

\begin{frame}
    \titlepage
\end{frame}

\begin{frame}{Contenido}
    \tableofcontents
\end{frame}


\section{Diseño: Preparación y Estructura}

\begin{frame}{1. Planificación: El Mensaje Central}
    \begin{itemize}
        \item \textbf{Objetivo Claro}: ¿Qué quiero que el público \textbf{sepa} o \textbf{haga}?
        \item \textbf{Mensaje Único}: El \textbf{corazón} de tu charla. (¡Una frase!)
        \item \textbf{Conoce a tu Audiencia}: Adapta el \textbf{lenguaje} y el \textbf{nivel}.
    \end{itemize}
    
    \vspace{0.5cm}
    \begin{figure}
        \centering
        \includegraphics[width=0.4\textwidth]{planificar.jpg} 
        \caption{Planificación Efectiva}
    \end{figure}
\end{frame}

\begin{frame}{2. Estructura: Flujo Lógico}
    \begin{itemize}
        \item \textbf{Introducción} ($\approx 10\%$): Gancho, tema, hoja de ruta.
        \item \textbf{Desarrollo} ($\approx 80\%$): Puntos clave, transiciones suaves.
        \item \textbf{Conclusión} ($\approx 10\%$): Resumen, llamada a la acción.
    \end{itemize}
    
    \vspace{0.5cm}
    \begin{figure}
        \centering
        \includegraphics[width=0.3\textwidth]{diagrama.pdf}
        \caption{Diagrama de Estructuración}
    \end{figure}
\end{frame}

\begin{frame}{3. Diseño Visual: Principios Clave}
    \begin{itemize}
        \item \textbf{Regla 6x6}: Máximo 6 líneas, máximo 6 palabras por línea. (¡Sé conciso!)
        \item \textbf{Contraste y Legibilidad}: Colores y fuentes \textbf{profesionales}.
        \item \textbf{Uso de Imágenes}: Relevantes, de \textbf{alta calidad}, y grandes.
    \end{itemize}
    
    \vspace{0.5cm}
    \begin{figure}
        \centering
        \includegraphics[width=0.5\textwidth]{mal_ejemplo.jpg}
        \caption{Lo que no hay que hacer}
    \end{figure}
\end{frame}

\section{Defensa: Ejecución y Habilidades Oratorias}

\begin{frame}{4. La Entrega: Voz y Lenguaje Corporal}
    \begin{itemize}
        \item \textbf{Contacto Visual}: Conecta con la \textbf{audiencia}.
        \item \textbf{Volumen y Ritmo}: Proyecta la voz, evita la \textbf{monotonía}.
        \item \textbf{Gestos Naturales}: Refuerzan el mensaje, evita \textbf{distracciones}.
    \end{itemize}
    
    \vspace{0.5cm}
    \begin{figure}
        \centering
        \includegraphics[width=0.45\textwidth]{lenguaje_no_verbal.png}
        \caption{La importancia del lenguaje corporal}
    \end{figure}
\end{frame}

\begin{frame}{5. Manejo del Tiempo y Ayudas Visuales}
    \begin{itemize}
        \item \textbf{Practica}: Ensaya para respetar los \textbf{tiempos} asignados.
        \item \textbf{Notas Mínimas}: El *frame* es la \textbf{guía}, no el guion.
        \item \textbf{Interacción}: Usa punteros/láser de forma \textbf{inteligente}.
    \end{itemize}
    
    \vspace{0.5cm}
    \begin{figure}
        \centering
        \includegraphics[width=0.3\textwidth]{tiempo.jpg}
        \caption{La gestión del tiempo}
    \end{figure}
\end{frame}

\begin{frame}{6. Sesión de Preguntas y Respuestas}
    \begin{itemize}
        \item \textbf{Prepárate}: Anticipa \textbf{preguntas difíciles} y respúndelas brevemente.
        \item \textbf{Escucha Activa}: Asegúrate de \textbf{entender} la pregunta.
        \item \textbf{Mantén la Calma}: Sé profesional, incluso ante la \textbf{crítica}.
    \end{itemize}
    
    \vspace{0.5cm}
    \begin{figure}
        \centering
        \includegraphics[width=0.3\textwidth]{preguntas.jpg}
        \caption{Prepárate para cualquier pregunta}
    \end{figure}
\end{frame}

\begin{frame}{Conclusión}
    \begin{itemize}
        \item \textbf{Diseño Impactante}: Simpleza y claridad visual.
        \item \textbf{Entrega Poderosa}: Confianza y conexión con la audiencia.
        \item \textbf{La Práctica es Clave}: ¡Ensaya, ensaya, ensaya!
    \end{itemize}
    
    \vspace{0.5cm}
    \begin{figure}
        \centering
        \includegraphics[width=0.4\textwidth]{exito.jpg}
        \caption{Ahora es tu turno de presentar con éxito}
    \end{figure}
\end{frame}


\end{document}