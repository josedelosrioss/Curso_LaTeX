\documentclass{beamer}
\usetheme{Frankfurt}
% \usecolortheme{beaver}

\usepackage[utf8]{inputenc}
\usepackage[spanish]{babel}

\title{Presentaciones creadas con el paquete Beamer}
\author{Tu mismo}
\date{\today}
\logo{\includegraphics[width=1.5cm]{Graficos/Curso-LaTeX/Graficos/graficos/ugr.png}}

\begin{document}
%
\begin{frame}
\titlepage
\end{frame}
%
\begin{frame}
\frametitle{Índice de la presentación}
\tableofcontents
\end{frame}
%
\section{Organización de contenidos usando Beamer}
%
% Incluir frames organizados en secciones y/o subsecciones
%
\begin{frame}{Beamer admite escritura en \LaTeX\ y más:}
\begin{enumerate}
\item Fórmulas: 
$\max\left\{2^{x+y}\int_a^b e^{\frac{x^2}{2}}\lim_{x\to 1}x^{x-1},1\right\}.$
\item Resultados Matemáticos:
\begin{theorem}\label{tonto}
Las ranas son verdes.
\end{theorem}
\item tablas
\begin{tabular}{|l|c|c|c|} \hline
Posición & 1 & 2 & 3 \\ \hline \hline
Nombre & Pepe & Juan & Manuel\\ \hline
\end{tabular}
\item y cualquier otro contenido...
\end{enumerate}
\end{frame}

\begin{frame}
\frametitle{Encapsulado de contenidos en bloques}
\begin{block}{Bloque normal}
Texto en un bloque normal
\end{block}
%
\begin{exampleblock}{Ejemplo}<uncover@2>
Bloque con un ejemplo: $f(x) = x^2$
\end{exampleblock}
%
\begin{alertblock}<3->{Advertencia}
¡Ojo con dividir por cero!
\end{alertblock}
\end{frame}
\end{document}