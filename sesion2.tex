\documentclass{article}

% Indicamos que el documento va a escribirse en castellano
\usepackage[spanish]{babel}

% Cargamos un paquete para poder incluir gráficos
\usepackage{graphicx}

\usepackage{amssymb}
\usepackage{amsmath}
\usepackage{mathtools}
% Activamos los hipervínculos del documento
\usepackage[hidelinks]{hyperref}

\DeclareMathOperator{\dist}{distancia}

% Añadimos la información sobre título autoría 
\title{Taller de \LaTeX{} para alumnos del Grado en Matemáticas}
\author{Los asistentes a dicho curso}
\date{\today}

\begin{document}
% Imprime la información del título proporcionada antes
\maketitle

\tableofcontents
\bigskip

Sea $f(x)=\cos(x)+\frac{1}{x}$ una función continua y sea $h(x) = \frac{1}{y}$. Si queremos una fórmula centrada podemos usar varias construcctiones:
\[
    h(x) = \frac{x}{y^2} .
\]
Consideramos la circunferencia
\begin{equation} \label{eq:circunferencia}
    x^2 + y^2 = 1 .
\end{equation}
La circunferencia de la ecuación~\eqref{eq:circunferencia} tiene radio 1.

Sean $f(x)=\frac{x}{2}$ y $\displaystyle
g(x)=\frac{x}{3}$ las funciones.

Sean $x_{1},x_{2},x_{3},\dots, x_{n}$ una sucesión de números reales.

\[
f(x) = 1+x, \quad (x>0)
\]

\[
\lim_{x \to \infty} f(x) = \infty
\]

\[
\sum_{\substack{i=1\\j=123}}^{n} \dfrac{i+j}{n} \frac{1}{x}
\]

\[
  \left(
  \begin{array}{lc|r}  % 3 columnas izq., centrada, derecha
    2 & 3 & \cos(x) \\
    -1 & 1 & 0 \\
    2 & 1 &  \\
  \end{array}
  \right)
\]

\[
  f(x) =
  \begin{cases}
   1+x^2, & \text{si $x<0$,}\\
   e^x,  & \text{si $x>0$,}\\
   1, & \text{si $x=0$.}
\end{cases}
\]

\[
  \begin{matrix}
  a_{11} & a_{12} & a_{13} \\
  a_{21} & a_{22} & a_{23}
  \end{matrix}
\]

\[
  \begin{pmatrix}
  a_{11} & a_{12} & a_{13} \\
  a_{21} & a_{22} & a_{23}
  \end{pmatrix}
  \quad
  \begin{bmatrix}
  a_{11} & a_{12} & a_{13} \\
  a_{21} & a_{22} & a_{23}
  \end{bmatrix}
  \quad
  \begin{Bmatrix}
  a_{11} & a_{12} & a_{13} \\
  a_{21} & a_{22} & a_{23}
  \end{Bmatrix}
\]

\[
f(x) =
\begin{dcases}
1+x^2, & \text{si $x<0$,}\\
\frac{x\sqrt{1+x^2}}{x^2+y^2}, & \text{si } x<0, \\
2, & \text{en otro caso}
\end{dcases}
\]

La matriz $A =
  \left(
  \begin{smallmatrix}
  1 & 2 \\
  3 & 4
  \end{smallmatrix}
  \right)$ se ve bien en una línea

\[
 \overbrace{a+\underbrace{b+c}_{z}+d}^{n}
\]

\[
\dist
\]

\begin{gather*}
    a+b=c+d \\
    z+x=y
\end{gather*}

\begin{align*}
x+y & = 6 \\
2x-3y & = 4
\end{align*}

\begin{equation}\label{eq:gather}
    f(x) = \left\{
        \begin{gathered}
            x+y+z+f(x)+1+2+3+4 \\
            +5+6+7+8+9+10
        \end{gathered}
        \right\}
        \iff
        \begin{aligned}
            f(x) & = x+y+z+f(x) \\
                    & \leq x^{32}+\cos{x} \\
                    & = \sum_{i=1}^{n} x_i
        \end{aligned}
\end{equation}

\end{document}