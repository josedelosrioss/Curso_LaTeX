\documentclass{article}

% Indicamos que el documento va a escribirse en castellano
\usepackage[spanish]{babel}

% Cargamos un paquete para poder incluir gráficos
\usepackage{graphicx}

\usepackage{amssymb}
\usepackage{amsmath}
\usepackage{mathtools}
% Activamos los hipervínculos del documento
\usepackage[hidelinks]{hyperref}

% Añadimos la información sobre título autoría 
\title{Taller de \LaTeX{} para alumnos del Grado en Matemáticas}
\author{Los asistentes a dicho curso}
\date{\today}

\begin{document}
% Imprime la información del título proporcionada antes
\maketitle

\tableofcontents
\bigskip

Sea $f(x)=\cos(x)+\frac{1}{x}$ una función continua y sea $h(x) = \frac{1}{y}$. Si queremos una fórmula centrada podemos usar varias construcctiones:
\[
    h(x) = \frac{x}{y^2} .
\]
Consideramos la circunferencia
\begin{equation} \label{eq:circunferencia}
    x^2 + y^2 = 1 .
\end{equation}
La circunferencia de la ecuación~\eqref{eq:circunferencia} tiene radio 1.

Sean $f(x)=\frac{x}{2}$ y $\displaystyle
g(x)=\frac{x}{3}$ las funciones.

Sean $x_{1},x_{2},x_{3},\dots, x_{n}$ una sucesión de números reales.

\[
f(x) = 1+x, \quad (x>0)
\]


\end{document}