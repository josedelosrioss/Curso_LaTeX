\documentclass{article}

\usepackage[spanish]{babel}
\usepackage{graphicx}
\usepackage{amssymb}
\usepackage{amsmath}
\usepackage{amsthm}
\usepackage{mathtools}
\usepackage{lipsum}
\usepackage[hidelinks]{hyperref}
\usepackage{tikz}
\usetikzlibrary{babel,shapes,arrows}

\title{Sesión 3. Gráficos}
\author{José de los Ríos Piñero}
\date{\today}

\begin{document}
\maketitle
\tableofcontents
\newpage
Test
%%%%%%%%%%%%%%%%%%%%%%%%%%%%%%%%%%%%%%%%%%%%%%%%%%%%%%%%%%%%%%%%%%%%%%%%%%%
\begin{figure}[h]
    \centering
    \includegraphics{Graficos/Curso-LaTeX/Graficos/graficos/ciclista.png}
    \caption{Un ciclista en acción}
    \label{fig:ciclista}
\end{figure}
%%%%%%%%%%%%%%%%%%%%%%%%%%%%%%%%%%%%%%%%%%%%%%%%%%%%%%%%%%%%%%%%%%%%%%%%%%%

\lipsum{1}

\begin{figure}[h]
    \centering  
    \includegraphics[angle=270, width=0.45\textwidth]{Graficos/Curso-LaTeX/Graficos/graficos/fig_9.pdf}
    \put(-50,-120){
        \includegraphics[scale=0.2, trim=10mm 15mm 8mm 5mm, clip]{Graficos/Curso-LaTeX/Graficos/graficos/ciclista.png}
    }
    \caption{Un ciclista en acción editado}
    \label{fig:ciclista_editado}
\end{figure}

Como se puede ver en la figura~\ref{fig:ciclista_editado}, el ciclista está en plena acción.
%%%%%%%%%%%%%%%%%%%%%%%%%%%%%%%%%%%%%%%%%%%%%%%%%%%%%%%%%%%%%%%%%%%%%%%%%%%
\section{Gráficos con TikZ}
\begin{figure}[h]
    \centering
    \begin{tikzpicture}
    \draw[help lines] (0,0) grid (3,3);
    \draw[->] (0,0) -- (2,2.5);
    \end{tikzpicture}
    \caption{Gráfico sencillo con TikZ}
\end{figure}
\end{document}
