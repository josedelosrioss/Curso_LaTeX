\documentclass{article}

% Indicamos que el documento va a escribirse en castellano
\usepackage[spanish]{babel}

% Cargamos un paquete para poder incluir gráficos
\usepackage{graphicx}

\newcommand{\parcial}[2]{
\frac{\partial #1}{\partial #2}
}

\usepackage{amssymb}
\newcommand{\R}{\mathbb{R}}

\newenvironment{nota}{
% Antes del entorno
\small \sffamily
\textbf{Nota: }
}
% Después del entorno
{
    \textbf{Fin de la nota\hfill \square}
}

% Activamos los hipervínculos del documento
\usepackage[hidelinks]{hyperref}

% Añadimos la información sobre título autoría 
\title{Taller de \LaTeX{} para alumnos del Grado en Matemáticas}
\author{Los asistentes a dicho curso}
\date{\today}

\begin{document}
% Imprime la información del título proporcionada antes
\maketitle

\tableofcontents

\begin{abstract}
\begin{abstract}
\textsf{Texto en sans-serif.} 

{\tiny Probando} {\scriptsize tamaños} {\footnotesize{en}} {\small esta} {\normalsize frase,} {\large esto} {\Large se} {\LARGE está} {\huge poniendo} {\Huge GRANDE}

\texttt{Texto monoespaciado}. \textbf{Texto en negrita.} \textit{Texto en cursiva.}

    Texto de ejemplo para el resumen: este es un párrafo de relleno que sustituye al uso del paquete lipsum y sirve para mostrar el formato del abstracto.
\end{abstract}
\section{Primera sección \Huge{Huge}}



Cuerpo del documento. Los       espacios en        blanco  y 
saltos de línea      no cuentan. 


Una o más líneas en blanco comienzan un nuevo párrafo.
una línea en blanco. 

También podemos escribir fórmulas en línea con el texto: $a^2+b^2=c^2$; o en bloque:
\[
f(x)=\cos(x)+\frac{1}{x}
\]

\subsection{Listas}
\begin{enumerate}
    \item Primer elemento
    \item Segundo elemento
    \begin{enumerate}
        \item Me
        \item Estoy
        \item Mareando
        \begin{enumerate}
            \item No
            \item Aguanto
            \item Más
        \end{enumerate}
    \end{enumerate}
    \item Tercer elemento
\end{enumerate}

\begin{itemize}
    \item[$\bullet$] Manzanas
    \item Peras
    \item Plátanos
\end{itemize}

\begin{description}
    \item[LaTeX] Lenguaje de programación para composición de textos científicos.
    \item[TeXStudio] Editor de \LaTeX 
\end{description}

Como podemos ver en el Cuadro \ref{table:datos-tabulados} en la página \pageref{table:datos-tabulados}

\[
\parcial{F}{x}
\]

\[
\R
\]

\begin{nota}
    Esto es una nota
\end{nota}

\subsection{Tablas}\label{subsec:tablas}
\begin{table}[htbp]
    \centering
    \begin{tabular}{|c|c|c|}
        \hline
        1 & 2 & 3 \\
        \hline
        Pepe & Juan & Manuel \\
        \hline
    \end{tabular}
    \caption{Datos tabulados}
    \label{table:datos-tabulados}

\end{table}

\begin{tabular}{lcr}
     1 & 2 & 3 \\
     Pepe & Juan & Manuel
\end{tabular}

\section{Segunda sección}
\section{Segunda sección}
Sección \ref{subsec:tablas}

Aquí va un párrafo de ejemplo que reemplaza la llamada a \lipsum; contiene varias oraciones para simular contenido real y mostrar el espaciado y formato del documento.
\subsection{Subsección}
\lipsum[1]


\end{document}